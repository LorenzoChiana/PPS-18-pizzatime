Per quanto riguarda le utilities ho sviluppato le seguenti:

\subsubsection{Action}
Rappresenta i tipi di azioni che può compiere il giocatore, ovvero il movimento e lo sparo.

\subsubsection{Difficulty}
Rappresenta le varie difficoltà che il gioco può avere, contiene una serie di parametri utilizzati all'interno del gioco. \textit{Difficulty} è stata poi ampliata, da me e dai miei colleghi, durante tutto lo svolgimento del progetto, per introdurre nuovi parametri.

\subsubsection{ImageType}
Rappresenta un enumerazione contenente tutti i tipi di sprite utilizzati all'interno del gioco con i relativi path.

\subsubsection{SoundType}
Come ImageType, rappresenta un enumerazione contenente tutti i tipi di suoni utilizzati all'interno del gioco con i relativi path.