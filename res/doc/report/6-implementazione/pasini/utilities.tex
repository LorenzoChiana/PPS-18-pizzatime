Nello sviluppare le cose sopracitate ho dovuto creare classi di utilità facenti parte del package Utilities, tra cui:
\begin{itemize}
    \item \textbf{Difficulty}: Enumerazione che rappresenta le difficoltà da scegliere e i relativi parametri. L'implementazione dell'enumerazione e la compilazione dei parametri sono comunque state oggetto di refactoring insieme agli altri membri durante lo sviluppo.
    \item \textbf{Direction}: Combinazione di sealed trait e case object per enumerare le possibili direzioni in cui un'entità può muoversi o essere rivolta.
    \item \textbf{Position}: La case class Position(point: Point, dir: Option[Direction]) rappresenta la posizione di un'entità nell'arena;
    \item \textbf{Point}: La case class Point(x: Int, y: Int) rappresenta un punto in 2D e serve anche a comporre Position.
    \item \textbf{Range}: La case class Range(min: Int, max: Int) rappresenta i range di valori ammissibili per la generazione delle entità in MapGenerator.
    \item \textbf{ImplicitConversions}: Object che contiene i metodi di conversione impliciti per semplificare la scrittura del codice. Il più importante ed utilizzato permette di poter instanziare un Point scrivendo una semplice tupla (x, y).
\end{itemize}