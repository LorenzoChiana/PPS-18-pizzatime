Durante soprattutto la fase iniziale del progetto ci siamo suddivisi i task da svolgere per sprint cercando di mantenere coerenza con i vari ambiti per essere più veloci.
D'altro canto, per poterci aiutare a vicenda è capitato un po' per tutti di andare a toccare il codice di altri componenti, finendo così ad avere un codice nel quale hanno messo le mani un po' tutti.
Questo ci ha dato una buona conoscenza di tutto il progetto.
Inoltre, per i task più complessi, si è deciso di optare per la tecnica di \textit{pair programming}.