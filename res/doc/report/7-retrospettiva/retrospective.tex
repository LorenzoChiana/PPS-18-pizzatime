Analizzando il nostro percorso di sviluppo, ci siamo accorti che si potrebbero introdurre migliorie sia per quanto riguarda la parte implementativa che per quanto riguarda quella organizzativa.

\subsubsection{Punto di vista organizzativo}
Nonostante abbiamo fatto una buona pianificazione, seguendo la metodologia scrum ed utilizzando Trello, qualche volta non sono state rispettate le linee guida previste per la sprint. Ovvero, a volte non sono state rispettate le urgenze indicate nei task in quanto venivano svolti prima task con un grado di urgenza inferiore.

Inoltre, sarebbe stato più appropriato utilizzare una metodologia git-flow basata su fork/pull request poichè, in questo modo si sarebbe introdotto un ulteriore controllo sulla qualità del codice caricato all'interno del repository.

\subsubsection{Punto di vista implementativo}
Avendo inizialmente sottovalutato la complessità dei calcoli presenti all'interno del gamelogic, è stato scelto un approccio di tipo single thread. Guardando lo stato finale del progetto, la complessità del gameLogic è salita, quindi sarebbe stato meglio se le computazioni al suo interno fossero state sviluppate concorretemente. In questo modo, si sarebbero evitati, probabilmente, anche i problemi nati con l'introduzione del codice Prolog.

