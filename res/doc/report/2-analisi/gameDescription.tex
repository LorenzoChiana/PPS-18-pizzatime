Il progetto consiste nello sviluppo di un gioco di tipo battle arena con grafica 2D a prospettiva fissa.

Il giocatore interpreta un personaggio che si muove all'interno di una mappa generata casualmente, sconfiggendo nemici e completando livelli.

Inoltre, il giocatore, può spostarsi da una stanza di dimensioni fisse ad un'altra mentre avanze di livello.
Una stanza può contenere un numero arbitrario di nemici, ostacoli e potenziamenti;
sconfiggere tutti i nemici garantisce l'accesso al livello successivo.

L'obiettivo del gioco potrebbe essere quello di raggiungere una stanza finale contenente un oggetto speciale che ne determina la fine, o, in alternativa, il completamento di un certo numero di stanze.
Nel primo caso, anche la probabilità di trovare l'oggetto speciale è casuale, ma dipende comunque da alcune condizioni di gioco come, ad esempio, il raggiungimento di una soglia di punteggio.
Tuttavia, il gioco potrebbe essere giocato in una modalità simile alla sopravvivenza solo per ottenere un punteggio personale più alto relativo all'utente, che viene mantenuto salvato e mostrato nella sezione della classifica del menu.
La natura casuale del gioco implica una curva di difficoltà facilmente variabile, facendo dipendere il gameplay anche dalla fortuna, che può renderlo divertente ma a volte anche impegnativo.

L'arma principale del giocatore sono i proiettili (pomodori) da lanciare ai nemici; gli oggetti da collezione garantiscono un bonus relativo al punteggio o alla vita.

Il gioco consente di registrare diversi profili utente per salvare le statistiche di gioco e confrontarle in una classifica.