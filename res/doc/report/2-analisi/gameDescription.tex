Il progetto consiste nello sviluppo di un gioco di tipo Battle Arena con grafica 2D a visuale fissa. Il giocatore interpreta un personaggio che si muove all'interno di una mappa generata casualmente, sconfiggendo nemici e completando livelli.

Il giocatore si sposta da una stanza di dimensioni fisse ad un'altra, che rappresentano i diversi livelli del gioco.
Una stanza può contenere un numero arbitrario di nemici, oggetti collezionabili e ostacoli; sconfiggere tutti i nemici garantisce l'accesso al livello successivo. Solo una stanza è visibile a schermo in un dato momento.

L'obiettivo del gioco potrebbe essere quello di raggiungere una stanza contenente un oggetto speciale che determina la fine della partita, o, in alternativa, il completamento di un certo numero di stanze.
Nel primo caso, anche la probabilità di trovare l'oggetto speciale è casuale, ma comunque dipendente da alcune condizioni di gioco (come, ad esempio, il raggiungimento di una soglia di punteggio).
Tuttavia, il gioco potrebbe essere giocato in una modalità che ha come scopo la sopravvivenza, solo per ottenere un punteggio personale più alto, che viene mantenuto salvato e mostrato nella sezione della classifica del menu.
La natura casuale del gioco implica una curva di difficoltà facilmente variabile, facendo dipendere il gameplay anche dalla fortuna, che può renderlo divertente ma a volte anche impegnativo.

L'arma principale a disposizione del giocatore sono proiettili (pomodori) da lanciare ai nemici; gli oggetti collezionabili garantiscono un bonus di punteggio o di vita.

Il gioco consente di registrare diversi profili utente per salvare le statistiche di gioco e confrontarle in una classifica.