Prima e durante la fase di sviluppo è stato necessario valutare delle scelte di implementazione, principalmente per cercare di sfruttare al meglio le caratteristiche di Scala, in modo da scrivere del codice intuitivo, compatto e scalabile, cercando anche di trarre il meglio dal paradigma funzionale e di garantire il più possibile immutablità e assenza di side-effect; un altro motivo di scelta principale ha riguardato anche la pesantezza del codice, avendo a che fare con un'applicazione particolarmente dinamica e reattiva. Di seguito sono riportate le scelte più significative che sono state adottate in fase di sviluppo:

\begin{itemize}
    \item \textbf{Case class}: Una scelta fondamentale, in quanto legata all'immutabilità del codice, ha riguardato l'utilizzo delle case class, in particolare per le entità. Queste sono caratterizzate da una posizione che per molte di esse cambia ad ogni step, ma volendo mantenerne lo stato immutabile, si è puntato a dichiararle come case class, in modo da poterle facilmente istanziare nuovamente a seguito di una modifica dello stato, nonchè per la facilità d'uso all'interno dei costrutti match-case. La gestione dello stato delle entità in questo modo è stata anche oggetto di refactoring durante lo sviluppo.
    
    Altra questione non meno importante riguarda la dichiarazione delle strutture dati nella classe \textit{Arena} (ma anche nel resto del programma), che possono potenzialmente rappresentare il collo di bottiglia per la velocità di esecuzione complessiva dell'applicazione. Seguendo la stessa logica utilizzata per le case class, sono sempre state usate strutture immutabili, dichiarate come var per garantirne la sostituzione in seguito ad un loro aggiornamento. L'immutabilità in questo caso assicura la coerenza nello stato del model.

    \item \textbf{GameLoop}: Il loop di controllo è stato oggetto di rifattorizzazione per quanto riguarda la struttura del ciclo e l'implementazione del suo thread. Il ciclo inizialmente seguiva una logica ricorsiva ed è stato poi convertito in un più semplice while-loop, per motivi meramente legati alla velocità di esecuzione, essendo un punto cruciale per il flusso logico del programma; la classe, invece, è passata dall'estendere \textit{Thread} all'implementare \textit{Runnable}, volendo avere una concorrenza basata sui task.
\end{itemize}